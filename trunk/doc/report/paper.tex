\documentclass[11pt]{article}
\usepackage{cite}
\usepackage{url}

\title{Discovery and Use Optimizations for Regular Stride Patterns in Compiler Prefetching}

\author{Patryk Mastela, Jason Varbedian, Matt Viscomi \\
  \small {\texttt {\{pmastela, jpvarbed, mviscomi\}@umich.edu}}}

\date{\today}

\begin{document}
  \maketitle

  \begin{abstract}
    - prefetching is hard
    - we reimpliment the profiling and prefetching
    - we also impliment some additional optimizations
    - these optimizations produce some kind of performance gains
    - if the gains are stable across then we can suggest this being used in a production compiler (this is word for word from ) ~\cite{WuEtAl2002}
  \end{abstract}
  
  \section{Introduction}
  Irregular programs contain many irregular data references, e.g., pointer-chasing code that manipulates dyanmic datat structures~\cite{WuEtAl2002}. Because of the irregularity of data references it is hard for both the compiler and hardware to anticipate the future address of a memory location~\cite{luk99}~\cite{roth98}. Others including Collins el al~\cite{collins01}, Stoutchinin et al~\cite{stoutchinin01}, and Wu et al~\cite{WuEtAl2002} have also taken
  notice that important loads in benchmark suite programs, e.g., {\textit 181.mcf} and {\textit 197.parser} in CPU2000, have near-constant strides. 
  \subsection{this is an intro subsection}
  
  \section{Experiments}
  tada citationg ~\cite{Wu2002} ~\cite{WuEtAl2002}
  ~\cite{metcalf93} ~\cite{prefetchsupportwebsite} ~\cite{dundas97} ~\cite{mowry98} ~\cite{luk99}
  \subsection{this is an intro subsection}

  \section{Conclusion}
  Restate the introduction.
  
  TODO hardware prefetch 
      its hard to show improvement. a lot o the easy prefetching is hard to do because if we prefetch too far than we're screwed and if too close we're screwed

  \bibliography{paper}{}
  \bibliographystyle{plain}
\end{document}
