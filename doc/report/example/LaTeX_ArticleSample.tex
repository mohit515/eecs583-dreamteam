

%% 
%% This is a LaTeX document generated by automatic conversion of a
%% notebook-format document using Publicon 1.0.
%% 


\documentclass{article}

\usepackage{graphicx,amscd,amsmath,amssymb,verbatim}
\usepackage[dvips]{hyperref}
\usepackage[TS1,OT1,T1]{fontenc}



\begin{document}


\title{Some Unrelated Topics of Interest}
\author{Michael Trott}
\date{\today}
\maketitle
\begin{abstract}

This paper offers a demonstration of the appearance and general
document features of the LaTeX\_Article style sheet. Although the
topics in this paper were written specifically for {\itshape Publicon},
it is hoped they will nonetheless be regarded with interest by the
physics community. {\copyright} 2004 Wolfram Research, Inc.
\end{abstract}
\section{High-precision Value for the Quartic Anharmonic Oscillator
Ground State}
\subsection{Introduction}

\noindent As is well known, only a very limited number of one--dimensional
potentials allow for an exact solution of the Schr\"odinger equation\footnote{This
footnote was not added by the author, but was placed for the purpose
of demonstration.}. This means that for many model potentials one
has to resort to numerical solution methods. For judging their accuracy,
reliability, and speed it is important to have high--precision values
of certain nonexactly solvable potentials. The most investigated
of such potentials is the quartic anharmonic oscillator \cite{{Tymczak,
Graffi, Sergeev, Suzuki, Chebot, Shanley, Hatsuda, Weniger, Ho96,
Cizek, Bay}}\cite{{Voros99, Khan, Antons, Skala, Meiszn, Delaba,
Fernan, Vinett}}described by
\begin{equation}
-\psi _{k}^{\prime \prime }( z) +z^{4}\psi _{k}( z) =\varepsilon
_{k}\psi _{k}( z) %
\label{XRef-Equation-22712381}
\end{equation}

The eigenfunctions to the eigenvalues $\varepsilon _{k}$ decay exponentially
for $z\rightarrow \pm \infty $. 
\subsection{The Hill Determinant Method}

\noindent A classical method to solve Sturm--Liouville problems
of type \ref{XRef-Equation-22712381} is to calculate the eigenvalues
of a truncated version of the corresponding Hill determinant. Using
the harmonic oscillator basis $\phi _{n}( z) $ we write $\psi _{0}(
z) =\sum \limits_{k=0}^{\infty }\alpha _{k}\phi _{k}( z) $ where
\begin{gather}
-\phi _{k}^{\prime \prime }( z) +z^{2}\phi _{k}( z) =\epsilon _{k}\phi
_{k}( z) 
\\\phi _{n}( z) =\frac{1}{\sqrt{\sqrt{\pi }2^{n}n!}}e^{\frac{-z^{2}}{2}}H_{n}(
z) 
\end{gather}

Forming the matrix elements $h_{m,n}=\int _{-\infty }^{\infty }\phi
_{m}( z) (-\phi _{n}^{{\prime\prime}}( z) +z^{4}\phi _{n}( z) )dz$.
For $n\geq m$ we obtain
\begin{equation}
h_{m,n}=\begin{cases}
0 & n-m>4 \\
\begin{array}{l}
 2^{\frac{m-n}{2}-4} \sqrt{\frac{m!}{n!}} \left( 32 n \delta _{m,n-2}
\left( n-1\right) ^{2}+16 \left( n-3\right)  \left( n-2\right) 
n \delta _{m,n-4} \left( n-1\right) +\right.  \\
 \left. \ \ \ \ \ \ \ \ \ \ \ \ \ \ \ \ \ \ \ \ \ 4 \left( 2 n \left(
3 n+5\right) +5\right)  \delta _{m,n}+8 \left( n+1\right)  \delta
_{m,n+2}+\delta _{m,n+4}\right) 
\end{array} & \textup{else} \\
\end{cases}
\end{equation}

A rough estimation shows that one obtains about 0.2 digits per harmonic
oscillator state. So by taking into account the first 500 eigenstates
and carrying out the calculation with about five thousand digits
one obtains about 120 reliable digits for $\varepsilon _{0}$. (This
calculation takes about 20 minutes on a 2000 vintage workstation
using \textit{Mathematica} 4 \cite{Wolfram99}.) 
\begin{multline*}
\varepsilon _{0}=\\
1.06036209048418289964704601669266354551520872852897793321624524169594356304434442112689629\\
9134671703510546244358582525580982763829\ldots 
\end{multline*}

The Hill determinant approach allows in addition to the calculation
of the eigenvalues, the calculation of the eigenvectors. The following
graphic visualizes the matrix of eigenvectors of $(h_{m,n})_{1\leq
n,m\leq 100}$. The graphic shows that the lowest eigenfunctions
are quite similar to the harmonic oscillator eigenfunctions. Higher
states are complicated mixtures of harmonic oscillator states. The
overall ``checkerboard''--like structure results from the fact that
the contribution of the antisymmetric (symmetric) harmonic oscillator
states to the symmetric (antisymmetric) anharmonic oscillator states
is identical zero. The very high states are dominated by truncation
effects and do not correctly mimic the anharmonic oscillator states.
\begin{figure}[h]
\begin{center}
\includegraphics{LaTeX_ArticleSample.tex_gr1.eps}

\end{center}
\caption{The matrix of eigenvectors of $(h_{m,n})_{1\leq n,m\leq
100}$.}
\end{figure}
\subsection{The New Algorithm}

\noindent To get a very high--precision approximation of\ \ 
\begin{equation}
-\psi ^{{\prime\prime}}( x) +x^{4}\psi ( x) =\lambda  \psi ( x)
\end{equation}

\noindent we start with the series expansion 
\begin{equation}
\psi ( x) =y_{n}( x) =\sum \limits_{k=0}^{n}a_{k}( \lambda ) x^{k}
\end{equation}

For the ground state we choose (ignoring normalization) $\psi (
0) =1, \psi ^{\prime }( 0) =0$. For ``suitable chosen'' $x^{*}$
we then find high--precision approximations for the zeros of $y_{n}(
x^{*}) $ and $y_{n}^{\prime }( x^{*}) $. These zeros then bound
$\lambda _{0}$ from below and above. \ \ 

Using the differential equation one obtains the following recursion
relation for the $a_{k}( \lambda ) $:
\begin{equation}
a_{m}( \lambda ) \ \ =\frac{a_{m-6}( \lambda ) -\lambda  a_{m-2}(
\lambda ) }{m^{2}-m}
\end{equation}

For large $n$ ($n\rightarrow \infty $) we want the function $y_{n}(
x) $ to vanish as $x\rightarrow \infty $. For a $\lambda$ smaller
than the smallest possible $\lambda$, the function $y_{n}( x^{*})
$ will not have a zero, but the function $y_{n}^{\prime }( x^{*})
$ will have a zero for a certain $x^{*}$. For a $\lambda$ larger
than the smallest possible $\lambda$, the function $y_{n}( x^{*})
$ will have a zero, but the function $y_{n}^{\prime }( x^{*}) $
will not have a zero for a certain $x^{*}$. This allows to find
a bounding interval for $\lambda _{0}$. The next two graphics show
$y_{80}( x) $ and $y_{80}^{\prime }( x) $ for 10 equidistant values
for $\lambda$ from the interval $[1.05,1.08]$ to visualize this
bounding process. (For more details, see \cite{Trott00A}.)
\begin{figure}[h]
\begin{center}
\includegraphics{LaTeX_ArticleSample.tex_gr2.eps}

\end{center}
\caption{Figure 2: $y_{80}( x) $ and $y_{80}^{\prime }( x) $ for
10 equidistant values for $\lambda$ from the interval $[1.05,1.08]$.}
\end{figure}

It is straightforward to implement the calculation of the bounding
interval for $\lambda _{0}$ in {\itshape Mathematica} in a three--line
program (see \cite{Trott00}). Using \texttt{FindRoot} we calculate
high--precision values for the zeros of $y_{n}( \xi ) $ and $y_{n}^{\prime
}( \xi ) $.


{\ttfamily
$\begin{array}{l}
\text{$\lambda$Bounds}\text{[}\text{n\_}\text{,}\text{ }\text{$\xi$\_}\text{,}\text{
}\text{opts\_\_\_}\text{]}\text{ }\text{:=}\text{ }\\
\text{\ \ \ }\,\text{Function}\text{[}\text{f}\text{,}\text{ }\text{$\lambda$}\text{
}\text{/.}\text{ }\text{FindRoot}\text{[}\text{f}\text{[}\text{n}\text{,}\text{
}\text{$\lambda$}\text{,}\text{ }\text{$\xi$}\text{]}\text{ }\text{==}\text{
}\text{0}\text{,}\text{ }\\
\text{\ \ \ \ \ \ \ }\,\text{\{}\text{$\lambda$}\text{,}\text{ }\text{106}\,\text{/}\,\text{100}\text{,}\text{
}\text{107}\,\text{/}\,\text{100}\text{\}}\text{,}\text{ }\text{opts}\text{]}\text{]}\text{
}\text{/@}\text{ }\text{\{}\text{y}\text{,}\text{ }\text{yPrime}\text{\}}\text{;}\text{
}\end{array}$}



The calculation of $y_{n}( \xi ) $ and $y_{n}^{\prime }( \xi ) $
is also straightforward based on a recursive calculations of the
$a_{k}( \lambda ) $. 


{\ttfamily
$\begin{array}{l}
\text{y}\text{[}\text{n\_}\text{,}\text{ }\text{$\lambda$\_Real}\text{,}\text{
}\text{$\xi$\_}\text{]}\text{ }\text{:=}\text{ }\text{Module}\text{[}\text{\{}\text{a6}\text{,}\text{
}\text{a4}\text{,}\text{ }\text{a2}\text{,}\text{ }\text{ak}\text{,}\text{
}\text{$\sigma$}\text{\}}\text{,}\text{ }\\
\text{\ \ \ }\,\text{\{}\text{a6}\text{,}\text{ }\text{a4}\text{,}\text{
}\text{a2}\text{\}}\text{ }\text{=}\text{ }\text{\{}\text{1}\text{,}\text{
}\text{-}\,\text{$\lambda$}\,\text{/}\,\text{2}\text{,}\text{ }\text{$\lambda$}\,\text{\textasciicircum
}\,\text{2}\,\text{/}\,\text{24}\text{\}}\text{;}\text{ }\\
\text{\ \ \ \ }\,\text{$\sigma$}\text{ }\text{=}\text{ }\text{a6}\text{
}\text{+}\text{ }\text{a4}\,\text{*}\,\text{$\xi$}\,\text{\textasciicircum
}\,\text{2}\text{ }\text{+}\text{ }\text{a2}\,\text{*}\,\text{$\xi$}\,\text{\textasciicircum
}\,\text{4}\text{;}\text{ }\\
\text{\ \ \ \ }\,\text{Do}\text{[}\text{ak}\text{ }\text{=}\text{
}\text{a6}\text{ }\text{-}\text{ }\text{$\lambda$}\,\text{*}\,\text{(}\text{a2}\,\text{/}\,\text{(}\text{k}\,\text{*}\,\text{(}\text{k}\text{
}\text{-}\text{ }\text{1}\text{)}\text{)}\text{)}\text{;}\text{
}\\
\text{\ \ \ \ \ \ }\,\text{\{}\text{a6}\text{,}\text{ }\text{a4}\text{,}\text{
}\text{a2}\text{\}}\text{ }\text{=}\text{ }\text{\{}\text{a4}\text{,}\text{
}\text{a2}\text{,}\text{ }\text{ak}\text{\}}\text{;}\text{ }\\
\text{\ \ \ \ \ \ }\,\text{$\sigma$}\text{ }\text{=}\text{ }\text{$\sigma$}\text{
}\text{+}\text{ }\text{ak}\,\text{*}\,\text{$\xi$}\,\text{\textasciicircum
}\,\text{k}\text{,}\text{ }\text{\{}\text{k}\text{,}\text{ }\text{6}\text{,}\text{
}\text{n}\text{,}\text{ }\text{2}\text{\}}\text{]}\text{;}\text{
}\text{$\sigma$}\text{]}\end{array}$}




{\ttfamily
$\begin{array}{l}
\text{y}\text{[}\text{n\_}\text{,}\text{ }\text{$\lambda$\_Real}\text{,}\text{
}\text{$\xi$\_}\text{]}\text{ }\text{:=}\text{ }\text{Module}\text{[}\text{\{}\text{a6}\text{,}\text{
}\text{a4}\text{,}\text{ }\text{a2}\text{,}\text{ }\text{ak}\text{,}\text{
}\text{$\sigma$}\text{\}}\text{,}\text{ }\\
\text{\ \ \ }\,\text{\{}\text{a6}\text{,}\text{ }\text{a4}\text{,}\text{
}\text{a2}\text{\}}\text{ }\text{=}\text{ }\text{\{}\text{1}\text{,}\text{
}\text{-}\,\text{$\lambda$}\,\text{/}\,\text{2}\text{,}\text{ }\text{$\lambda$}\,\text{\textasciicircum
}\,\text{2}\,\text{/}\,\text{24}\text{\}}\text{;}\text{ }\\
\text{\ \ \ \ }\,\text{$\sigma$}\text{ }\text{=}\text{ }\text{a6}\text{
}\text{+}\text{ }\text{2}\,\text{*}\,\text{$\xi$}\,\text{*}\,\text{a4}\text{
}\text{+}\text{ }\text{4}\,\text{*}\,\text{$\xi$}\,\text{\textasciicircum
}\,\text{3}\,\text{*}\,\text{a2}\text{;}\text{ }\\
\text{\ \ \ \ }\,\text{Do}\text{[}\text{ak}\text{ }\text{=}\text{
}\text{a6}\text{ }\text{-}\text{ }\text{$\lambda$}\,\text{*}\,\text{(}\text{a2}\,\text{/}\,\text{(}\text{k}\,\text{*}\,\text{(}\text{k}\text{
}\text{-}\text{ }\text{1}\text{)}\text{)}\text{)}\text{;}\text{
}\\
\text{\ \ \ \ \ \ }\,\text{\{}\text{a6}\text{,}\text{ }\text{a4}\text{,}\text{
}\text{a2}\text{\}}\text{ }\text{=}\text{ }\text{\{}\text{a4}\text{,}\text{
}\text{a2}\text{,}\text{ }\text{ak}\text{\}}\text{;}\text{ }\\
\text{\ \ \ \ \ \ }\,\text{$\sigma$}\text{ }\text{=}\text{ }\text{$\sigma$}\text{
}\text{+}\text{ }\text{k}\,\text{*}\,\text{ak}\,\text{*}\,\text{$\xi$}\,\text{\textasciicircum
}\,\text{(}\text{k}\text{ }\text{-}\text{ }\text{1}\text{)}\text{,}\text{
}\text{\{}\text{k}\text{,}\text{ }\text{6}\text{,}\text{ }\text{n}\text{,}\text{
}\text{2}\text{\}}\text{]}\text{;}\text{ }\text{$\sigma$}\text{]}\end{array}$}



Calculating now \texttt{$\lambda$Bounds[16000, 16, }\textit{{\itshape
startingValues}}\texttt{, WorkingPrecision$ \rightarrow\ \ $6000,
AccuracyGoal$ \rightarrow\ \ $600, MaxIterations$ \rightarrow\ \ $100]}
(where \textit{startingValues} has been obtained from a call to
\texttt{$\lambda$Bounds} recursively, one gets in a few minutes
a 1184 digit approximation to the ground state energy of the quartic
anharmonic oscillator.  
\begin{multline*}
\varepsilon _{0}=\\
1.06036209048418289964704601669266354551520872852897793321624524169594356304434442112689629\\
91346717035105462443585825255808798082102931470131768363738249357892262460047081754469601\\
41637488417282256905935757790888061788790263601549395690275196148900942934873584409442694\\
89790121397146429095192335453382834703350575761511202570398885237202402218411030865737310\\
91398915453658410311167940583354860009227440069631126702388622971429699610592155832266713\\
76935508673610000831830027517926233573913906136180776498596961814994127928092728407079561\\
06044072294680994913627572927387279136890279842472226171694448895475137043806840543918778\\
77295323424587437254317832319060381068741604403437453014684727813918612940470431034013510\\
71607110353008929823275427661518986950565047160252756089526262191025688200964410287815640\\
05270529293240507638265028259112477362538471854714402572285438485297450458570978840249066\\
99957047684458770917620291243752732549071164334402302947306923981908956853745359884460160\\
02313291933059395869304916644281633946163324287004261461237743009952234204208597735690153\\
56541685030894185134879573410658547971946759646679661346768858643795265451956056828671595\\
8338884743467012042420714919290048732\ldots 
\end{multline*}

Statistical analysis of the number does not show any regularity.
\subsection{Summary}

\noindent A power series based approach to the high--precision calculation
of the ground state of the anharmonic oscillator was presented.
{\itshape Mathematica} code to carry out the calculation, as well
as results were given. The method can straightforwardly be used
to calculate ten thousands of digits of the quartic anharmonic,
as well as other anharmonic oscillators. Work concerning the application
of the method to higher states is in progress. 

All calculations and visualizations have been carried out in {\itshape
Mathematica} 4.
\subsection{Local Density of States for the Harmonic Oscillator}

\noindent The next graphic shows the local density of states ${\mathcal
D}_{E}( x) =\langle x|\theta ( E-\hat{H}) |x\rangle $ for the harmonic
oscillator \cite{Trott00A}. 
\begin{figure}[h]
\begin{center}
\includegraphics{LaTeX_ArticleSample.tex_gr3.eps}

\end{center}
\caption{Figure 3: The local density of states ${\mathcal D}_{E}(
x) =\langle x|\theta ( E-\hat{H}) |x\rangle $ for the harmonic oscillator.}
\end{figure}
\subsection{Acknowledgments}

\noindent The author would like to thank Andr\'e Kuzniarek for making
a prerelease version of the {\itshape Publicon} typesetting system.
This work was supported by Wolfram Research, Inc.
\section{Are Brillouin Zones of High Order Fractal?}
\subsection{Introduction}

\noindent Brillouin zones are among the most popular objects a solid
state physicist deals with \cite{{Brillo, Ashcro76, Crackw, Landsb}}.
Despite the fundamental importance for the explanation of most properties
of crystalline solids, Brillouin zones as an own subject have rarely
been investigated (the only ones we are aware of are \cite{{Bieber,
Jones, Skriga, Veerman}}). For electronic properties, mostly the
low order Brillouin zones matter, as an ownstanding subject the
high order Brillouin zones are interesting. Mathematically, the
$(n+1)$th Brillouin zone is the set of points that a line to them
crosses exactly $n$ bisector planes. In computational geometry a
$n$th order Brillouin zones is also called $n$th degree Voronoi
region or \textit{n}th nearest point Voronoi diagrams. The most
important fact for high order Brillouin zones is that their shape
approaches that of a thin spherical shell and their volume is a
constant. Here, for the first time we report on some computational
results of higher order Brillouin zones. All calculations and visualizations
were done with \textit{Mathematica} 4 \cite{Wolfram99}. 

Recursive definition of Brillouin zones: Given a lattice $\Lambda
$ in $\mathbb{R}^{d}$with lattice points $h_{i}$ ($i$ being a multiindex)
the first Brillouin zone ${\mathcal B}{\mathcal Z}_{1}$ is the closure
of the set of all points $x$ such that $|x-0|\leq $$|x-h_{i}|$ for
all $h_{i}${\bfseries $ \neq\ \ $}$0$. The $n$th order Brillouin
zones is the closure of the set of all points $x$ such that $|x-0|\leq
$$|x-h_{i}|$ for all $h_{i}$$ \neq\ \ $$0$ and $x\notin {\mathcal
B}{\mathcal Z}_{n-1}$. 
\subsection{2D Hexagonal Lattice}

\noindent Figure \ref{XRef-Figure-22712467} shows the first twenty
Brillouin zones of a 2D hexagonal lattice. It is interesting to
observe that the first, third, and fourth Brillouin zones have the
shape of an hexagon. For higher orders the shape becomes much more
complicated. 
\begin{figure}[h]
\begin{center}
\includegraphics{LaTeX_ArticleSample.tex_gr4.eps}

\end{center}\label{XRef-Figure-22712467}
\caption{Figure 4: The first twenty Brillouin zones of a 2D hexagonal
lattice.}
\end{figure}

Figure \ref{XRef-Figure-227124554} shows the 200th Brillouin zone
in one symmetry unit (inside an angle of $30^{\mbox{}^{\circ}}$).
One sees many small and a few quite large Landsberg zones. The distribution
$p( A) $ of the area of the Landsberg zones in the limit $n\rightarrow
\infty $ might be an interesting subject to study. 
\begin{figure}[h]
\begin{center}
\includegraphics{LaTeX_ArticleSample.tex_gr5.eps}

\end{center}\label{XRef-Figure-227124554}
\caption{Figure 5: The 200th Brillouin zone in one symmetry unit
(inside an angle of $30^{\mbox{}^{\circ}}$).}
\end{figure}

A good numerical fit to number of faces (line segments) $\sharp
_{n}$ of the $n$th Brillouin zone is $\sharp _{n}\propto n^{1.15}$.
Figure \ref{XRef-Figure-227124537} shows the circumference of the
Brillouin zones normalized to the circumference of a circle with
the same radius. Does the ratio approach a finite value in the limit
$n\rightarrow \infty $? 
\begin{figure}[h]
\begin{center}
\includegraphics{LaTeX_ArticleSample.tex_gr6.eps}

\end{center}\label{XRef-Figure-227124537}
\caption{Figure 6: The circumference of the Brillouin zones normalized
to the circumference of a circle with the same radius.}
\end{figure}
\subsection{3D Cubic Lattices}

\noindent In \cite{Trott00B} we gave a complete implementation for
the effective calculation of higher order Brillouin zones of the
three cubic lattice in $\mathbb{R}^{3}$. Figure \ref{XRef-Figure-227124524}
shows the (outside of) 15th Brillouin zone for the simple cubic,
Figure \ref{XRef-Figure-227124514} shows the 18th for the face--centered
cubic and Figure \ref{XRef-Figure-227124459} shows the 10th for
the body--centered cubic lattice. The higher order Brillouin zones
show quite complicated behavior. Large faces alternate with small
ones, relatively plane regions alternate with quite structured ones.
The appearance of the $n+1$th Brillouin zone is typically completely
independent of the appearance of the $n$th Brillouin zone. 
\begin{figure}[h]
\begin{center}
\includegraphics{LaTeX_ArticleSample.tex_gr7.eps}

\end{center}\label{XRef-Figure-227124524}
\caption{Figure 7: The (outside of) 15th Brillouin zone for the
simple cubic.}
\end{figure}
\begin{figure}[h]
\begin{center}
\includegraphics{LaTeX_ArticleSample.tex_gr8.eps}

\end{center}\label{XRef-Figure-227124514}
\caption{Figure 8: The (outside of) 18th Brillouin zone for the
face--centered cubic.}
\end{figure}
\begin{figure}[h]
\begin{center}
\includegraphics{LaTeX_ArticleSample.tex_gr9.eps}

\end{center}\label{XRef-Figure-227124459}
\caption{Figure 9: The (outside of) 10th Brillouin zone for the
body--centered cubic lattice.}
\end{figure}

Using this implementation we analysed various properties of higher
order Brillouin zones. Figure \ref{XRef-Figure-227124436} shows
the area of the Brillouin zones normalized to the area of a sphere
of the same volume. The order of the three curves from the bottom
is sc, fcc, bcc. 
\begin{figure}[h]
\begin{center}
\includegraphics{LaTeX_ArticleSample.tex_gr10.eps}

\end{center}\label{XRef-Figure-227124436}
\caption{Figure 10: The area of the Brillouin zones normalized to
the area of a sphere of the same volume.}
\end{figure}

The last result to be given here is the number of faces of the Brillouin
zones. By a face we mean any connected planar part of the outside
facing side of a Brillouin zone (point contacts separate faces).
Table \ref{XRef-Table-227124417} gives the results for the first
25 Brillouin zones.
\begin{table}
\caption{The results for the first 25 Brillouin zones.}
\begin{center}
\begin{tabular}{llll}
\textit{n} & \textit{sc} & \textit{fcc} & \textit{bcc}\\
\hline
1 & 6 & 14 & 12\\
2 & 12 & 72 & 48\\
3 & 72 & 96 & 30
\end{tabular}
\end{center}\label{XRef-Table-227124417}
\end{table}
\begin{table}
\caption{The results for the first 25 Brillouin zones.}
\begin{center}
\begin{tabular}{ll}
\textit{n} & \textit{sc}\\
\hline
1 & 6
\end{tabular}
\end{center}
\end{table}
\begin{table}
\caption{The results for the first 25 Brillouin zones.}
\begin{center}
\begin{tabular}{llll}
\textit{n} & \textit{sc} & \textit{fcc}\\
\hline
1 & 6 & 14\\
2 & 12 & 72
\end{tabular}
\end{center}
\end{table}
\begin{table}
\caption{The results for the first 25 Brillouin zones.}
\begin{center}
\begin{tabular}{lllll}
\textit{n} & \textit{sc} & \textit{fcc} & \textit{bcc} & \textit{bcc}\\
\hline
1 & 6 & 14 & 12 & 12\\
2 & 12 & 72 & 48 & 48\\
3 & 72 & 96 & 30 & 30
\end{tabular}
\end{center}
\end{table}
\subsection{Summary}

\noindent Preliminary results about some computational results about
higher order Brillouin zones have been presented. Further work is
in progress and will be published elsewhere.

All calculations and visualizations have been carried out in \textit{Mathematica}
4. 
\subsection{Acknowledgments}

The author would like to take the opportunity to thank Amy Young
for assistance in typesetting this paper.

\appendix

\section{Index Gymnastics}

\noindent The technique of extracting the content from geometric
(tensor) equations by working in component notation and rearranging
\href{http://mathworld.wolfram.com/Index.html}{indices} as required\footnote{$
<<Conversion\ Failed>>% URL["http://mathworld.wolfram.com"]
 $}. Index gymnastics is a fundamental component of \href{http://scienceworld.wolfram.com/physics/SpecialRelativity.html}{special}
and \href{http://scienceworld.wolfram.com/physics/GeneralRelativity.html}{general
relativity} \cite{Misner}. Examples of index gymnastics include
\begin{gather*}
S^{\alpha \beta }_{\gamma }=g^{\beta \mu }S^{\alpha }_{\mu \gamma
}
\\S^{\alpha }_{\mu \gamma }=g_{\mu \beta }S^{\alpha \beta }_{\gamma
}
\\A^{2}=A^{\alpha }A_{\alpha }
\\g_{\alpha \beta }g^{\beta \gamma }=\delta _{\alpha }^{\gamma }
\\N^{\alpha }_{\beta }^{,\gamma }=N^{\alpha }_{\beta ,\mu }g^{\mu
\gamma }
\\\left( R_{\alpha }M_{\beta }\right) _{,\gamma }=R_{\alpha ,\gamma
}M_{\beta }+R_{\alpha }M_{\beta ,\gamma }
\\F_{\left[ \alpha \beta \right] }=\frac{1}{2}\left( F_{\alpha \beta
}-F_{\beta \alpha }\right) 
\\F_{\left( \alpha \beta \right) }=\frac{1}{2}\left( F_{\alpha \beta
}+F_{\beta \alpha }\right) 
\end{gather*}

\noindent \cite{Misner}, where $g_{i j}$ is the \href{http://mathworld.wolfram.com/MetricTensor.html}{metric
tensor}, $\delta _{\alpha }^{\gamma }$ is the \href{http://mathworld.wolfram.com/KroneckerDelta.html}{Kronecker
delta}, , is a \href{http://mathworld.wolfram.com/CommaDerivative.html}{comma
derivative}, $F_{[\alpha \beta ]}$ is the \href{http://mathworld.wolfram.com/AntisymmetricTensor.html}{antisymmetric
tensor} part, and $F_{(\alpha \beta )}$ is the\href{http://mathworld.wolfram.com/SymmetricTensor.html}{
symmetric tensor} part\footnote{$ <<Conversion\ Failed>>% URL["http://mathworld.wolfram.com/Index.html"]
 $}.\label{foot}\label{mww}\label{seealsos}
\begin{thebibliography}{000}
\bibitem{Tymczak} Tymczak, C. J., Japaridze, G. S., Handy, C. R.,
\& Wang, X. Q. (1998, March 14). \textit{New perspective on inner
product quantization}. \textit{Phys. Rev. Lett.}, \textbf{80}(17),
3673--3677.\label{Tymczak}
\bibitem{Graffi} Graffi, S., \& Grecchi, V. (1973, November 15).
\textit{Rayleigh-Ritz Method, Secular Determinant, and Anharmonic
Oscillators}. \textit{Phys. Rev. D}, \textbf{8}(10), 3487--3492.\label{Graffi}
\bibitem{Sergeev} Sergeev, A. V., \& Goodson, D. Z. (1998, March
14). \textit{Bloch electrons in a magnetic field}. \textit{J. Phys.},
\textbf{A}(31), 4301--4301.\label{Sergeev}
\bibitem{Suzuki} Suzuki, J. (1999, March 14). \textit{A new ambient
pressure organic superconductor based on BEDT-TTF with 10.4-K higher
than 10K}. \textit{J. Phys.}, \textbf{A}(32), 183--187.\label{Suzuki}
\bibitem{Chebot} Chebotarev, L. V. (1997, Dec 12). \textit{Parabolic
connection formulae in quantum mechanics}. \textit{Ann. Phys.},
\textbf{255}(2), 305--332.\label{Chebot}
\bibitem{Shanley} Shanley, P. E. (1988, March 14). \textit{Large-order
analysis of the convergent renormalized strong-coupling perturbation
theory for the quartic anharmonic oscillator}. \textit{Ann. Phys.},
\textbf{186}(3), 292--325.\label{Shanley}
\bibitem{Hatsuda} Hatsuda, T., Kunihiro, T., \& Tanaka, T. (1997,
April 28). \textit{Optimized Perturbation Theory for Wave Functions
of Quantum Systems}. \textit{Phys. Rev. Lett.}, \textbf{78}(17),
3229--3230.\label{Hatsuda}
\bibitem{Weniger} Weniger, E. J. (1996, September 30). \textit{Construction
of the Strong Coupling Expansion for the Ground State Energy of
the Quartic, Sextic, and Octic Anharmonic Oscillator via a Renormalized
Strong Coupling Expansion}. \textit{Phys. Rev. Lett.}, \textbf{77}(14),
2859--2862.\label{Weniger}
\bibitem{Ho96} Ho, K. C., et al. (1996, March 12). \textit{Study
of quantum anharmonic oscillators by state-dependent diagonalization}.
\textit{Phys. Rev. A.}, \textbf{53}(3), 1280--1284.\label{Ho96}
\bibitem{Cizek} \v{C}izek, J., Weniger, E. J., Bracken, P., \& \v{S}piro,
V. (1996, Dec 12). \textit{Effective characteristic polynomials
and two-point Pad\'e approximants as summation techniques for the
strongly divergent perturbation expansions of the ground state energies
of anharmonic oscillators}. \textit{Phys.Rev.}, \textbf{53}(3),
2925--2939.\label{Cizek}
\bibitem{Bay} Bay, K., \& Lay, W. (1997, Dec 12). \textit{The spectrum
of the quartic oscillator}. \textit{J. Math. Phys.}, \textbf{38}(5),
2127--2131.\label{Bay}
\bibitem{Voros99} Voros, A. (1999, March 14). \textit{Magnetic field
dependence of the cyclotron effective mass in the organic superconductor}.
\textit{J. Phys.}, \textbf{32}(A), 5993--6000.\label{Voros99}
\bibitem{Khan} Khan, P. B., \& Zarmi, Y. (1999, Dec 12). \textit{Properties
of new organic conductors}. \textit{J. Math. Phys.}, \textbf{40}(1),
4658--4659.\label{Khan}
\bibitem{Antons} Antonsen, F. (1997, Dec 12). \textit{Quantum theory
in curved spacetime using the Wigner function}. \textit{Phys. Rev.
D.}, \textbf{56}(2), 920--935.\label{Antons}
\bibitem{Skala} Sk\'ala, L., \v{C}izek, J., Kapsa, V., \& Weniger,
E. J. (1997, December). \textit{Large-order analysis of the convergent
renormalized strong-coupling perturbation theory for the quartic
anharmonic oscillator}. \textit{Phys. Rev. A}, \textbf{56}(6), 4471--4476.\label{Skala}
\bibitem{Meiszn} Mei{\ss}ner, H., \& Steinborn, E. O. (1997, March
14). \textit{Large-order analysis of the convergent renormalized
strong-coupling perturbation theory for the quartic anharmonic oscillator}.
\textit{Phys. Rev.}, \textbf{56}(A), 4471--4476.\label{Meiszn}
\bibitem{Delaba} Delabaere, E., \& Pham, F. (1997, Dec 12). \textit{Unfolding
the quartic oscillator}. \textit{Ann. Phys}, \textbf{261}(2), 180--218.\label{Delaba}
\bibitem{Fernan} Fern\'andez, F., \& Guardiola, R. (1993, Dec 12).
\textit{Accurate eigenvalues and eigenfunctions for quantum-mechanical
anharmonic oscillators}. \textit{J. Phys. A.}, \textbf{26}(23),
7169--7180.\label{Fernan}
\bibitem{Vinett} Vinette, F., \& \v{C}izek, J. (1991, March 14).
\textit{Spinless fermions on frustrated lattices in a magnetic field}.
\textit{J. Math. Phys.}, \textbf{26}(3), 3392--3396.\label{Vinett}
\bibitem{Wolfram99} Wolfram, S. (1999). \textit{The }\textit{{\itshape
Mathematica}}\textit{ Book}. Champaign: Cambridge University Press
and Wolfram Media.\label{Wolfram99}
\bibitem{Trott00A} Trott, M. (2000). \textit{The }\textit{{\itshape
Mathematica}}\textit{ GuideBook: Mathematics in }\textit{{\itshape
Mathematica}}. New York: Springer-Verlag.\label{Trott00A}
\bibitem{Trott00} Trott, M. (2000). \textit{The }\textit{{\itshape
Mathematica}}\textit{ GuideBook: Programming in Mathematica}. New
York: Springer-Verlag.\label{Trott00}
\bibitem{Brillo} Brillouin, L. (1946). \textit{Wave Propogation
in Periodic Structures}. New York: McGraw-Hill.\label{Brillo}
\bibitem{Ashcro76} Ashcroft, N. W., \& Mermin, N. D. (1976). \textit{Solid
State Physics}. Philadelphia: Saunders College.\label{Ashcro76}
\bibitem{Crackw} Crackwell, A. P., \& Wong, K. C. (1973). \textit{The
Fermi Surface}. Oxford: Clarendon Press.\label{Crackw}
\bibitem{Landsb} Landsberg, P. T. (1969). \textit{Solid State Theory}.
London: Wiley.\label{Landsb}
\bibitem{Bieber} Bieberbach, L. (1939, Dec 12). \textit{\"Uber die
Inhaltsgleichheit der Brillouinschen Zonen}. \textit{Monatsh. Math.
Phys.}, \textbf{48}(1), 509--515.\label{Bieber}
\bibitem{Jones} Jones, G. A. (1984, Dec 12). \textit{An algorithm
for machine calculation of complex Fourier series}. \textit{Bull.
London Math. Soc.}, \textbf{16}(1), 241--242.\label{Jones}
\bibitem{Skriga} Skriganov, M. M. (1984, March 14). \textit{Mulliken-Wolfsberg-Helmholtz
band structure of di-tetramethyltetraselenafulvalene-}\textit{{\itshape
X}}\textit{ }\textit{$[(\operatorname{TMTSF})_{2}X]$}\textit{: Role
of the basis set}. \textit{Zap. Nautn. Sem.Leningrad. Otdel. Mat.
Inst. Steklov.}, \textbf{134}(3), 206--210.\label{Skriga}
\bibitem{Veerman} Veerman, J. J., et al. (2000, March 14). \textit{On
Brillouin zones}. \textit{Comm. Math. Phys.}, \textbf{ 212}(3),
725--744.\label{Veerman}
\bibitem{Trott00B} Trott, M. (2000). \textit{The }\textit{{\itshape
Mathematica}}\textit{ GuideBook: Graphics in Mathematica}. New York:
Springer-Verlag.\label{Trott00B}
\bibitem{Misner} Misner, C. W., Thorne, K. S., \& Wheeler, J. A.
(1973). \textit{Gravitation}. San Francisco: W. H. Freeman.\label{Misner}
\end{thebibliography}

\end{document}
